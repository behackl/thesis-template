\documentclass[11pt, twoside, withdegree]{bhthesis}

\usepackage[ngerman, english]{babel}

\newcommand{\OEIS}[1]{\text{\href{https://oeis.org/#1}{{\small \tt (#1)}}}}
\DeclarePairedDelimiter{\abs}{\lvert}{\rvert}
\DeclarePairedDelimiter{\iverson}{\llbracket}{\rrbracket}

\title{Eine supertolle Masterarbeit}
\author{Maria Musterfrau}

\reporttype{Masterarbeit}
\studname{Angewandte Informatik}
\degree{Diplom-Ingenieurin}

\involvedpeople{
  \person{0.45\linewidth}{Betreuerin}{
    \hbox{Univ.-Prof.\ Dr.\ Alexandra Musterfrau}\\
    Institut für Angewandte Informatik\\
    Alpen-Adria-Universität Klagenfurt
  }\hfill
  \person[\flushright]{0.45\linewidth}{Gutachter}{
    Dr.\ Erich Mustermann\\
    Institut für Mathematik\\
    Technische Universität Graz\\
  }\\[2em]
}


\university{
  \hfill\includegraphics{aau-logo.pdf}\\[1em]
}
\universityname{Alpen-Adria-Universität Klagenfurt}
\fakultaetname{Fakultät für Technische Wissenschaften}


\begin{document}
\selectlanguage{ngerman}
\maketitle

\selectlanguage{english}
\chapter*{Abstract}
This is a short summary of the contents of this thesis.

\tableofcontents

\chapter{Introduction}\label{chap:intro}

\section{Tasty!}

\begin{definition}[Open Set]
  A set $\Omega\subseteq\mathbb{C}$ is said to be \emph{open}, if for every
  $z_{0}\in\Omega$ there is a positive number $\varepsilon > 0$ such
  that for all $z\in\mathbb{C}$ that satisfy $\abs{z - z_{0}} <
  \varepsilon$ we find $z\in\Omega$.
\end{definition}

\begin{theorem}[Cauchy's Integral Theorem]\label{thm:cauchy}
  Let $\Omega\subseteq\mathbb{C}$ be an open and simply connected
  set. Let $\gamma:[0,1]\to \Omega$ be a closed path in $\Omega$, and
  let $f\colon\gamma^{*} \to \mathbb{C}$ be a holomorphic function.
  Then the relation
  \begin{equation}\label{eq:cauchy-integral}
    \oint_{\gamma} f(z)~dz = 0
  \end{equation}
  holds.
\end{theorem}

The following example provides verification for this important result.

\begin{example}
  Consider $f\colon \mathbb{C} \to \mathbb{C}$ with $z\mapsto
  z^{2}$. Then, by Theorem~\ref{thm:cauchy} we find
  \[ \oint_{\abs{z} = 42} f(z)~dz = 0.  \]
  In this case, it is not too difficult to verify that the theorem
  holds by straightforward computation of the line integral; we use
  the curve $\gamma\colon [0,1]\to \mathbb{C}$ with $\gamma(t) =
  42\cdot\exp(2\pi i t)$. Straightforward computation yields
  \begin{align*}
    \oint_{\abs{z} = 42} f(z)~dz
    &= \int_{0}^{1} f(\gamma(t)) \gamma'(t)~dt\\
    &= 42^{3}\cdot 2\pi i \cdot \biggl[\frac{\exp(6 \pi i t)}{6 \pi
      i}\biggr]_{0}^{1} = 42^{3}\cdot 2\pi i\cdot \frac{1 - 1}{6\pi
      i} = 0,
  \end{align*}
  which verifies the theorem.
\end{example}

\begin{theorem*}
  Theorems do not have to be numbered, but they can stretch over
  multiple lines and maybe even over to the next page. A very nice
  formula is
  \[ e^{\pi i} + 1 = 0, \]
  and it should help to illustrate a page break.
\end{theorem*}

\section{Improvements?}

Feel free to adapt / polish the styling suggested by this template
in any way you like. This template is hosted at
\url{https://github.com/behackl/thesis-template} -- I am happy to
discuss ideas and suggestions for general improvement of this template.



\end{document}